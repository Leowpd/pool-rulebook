\fancytitle[images/whiteball.png]{Appendices}

\renewcommand{\thesection}{\Alph{section}}

\section{Credits \& Version History} \label{credits}

\setlength{\parindent}{0pt}

The Pilaar Duncan Family Pool Rules are the house rules for pool developed by and for the Pilaar Duncan family, starting in Christmas 2016.\standardspace[large]
The rules were developed by (alphabetically) Chris Duncan, Leo Duncan, Lily Duncan, Tom Duncan, Carla Pilaar, \& Mark Pilaar.\standardspace[large]
This document was written by Leo Duncan.\standardspace[large]
Version 1 (2016) of these rules existed orally and in the minds of the players. Version 2 (2022) was a partially completed Google Doc. This is Version 3 (2024).\standardspace[large]
Current Version: \currentversion.\standardspace[large]
Last Revision: \currentversiondate.\standardspace[large]
Changes: {\small \url{https://github.com/Leowpd/pool-rulebook}}

\newpage

\section{Glossary} \label{glossary}

\small
\begin{center}
    \begin{tabular}{  p{0.145\textwidth}  p{0.291\textwidth}  } 
      \hline
      \textbf{Term} & \textbf{Definition} \\ \hline
      Black, The & Alternative term for the 8-ball. \\ \hline
      Clank, The & A mishit that causes a truly angelic “clank” sound. \\ \hline
      Down Trout & Mispronunciation of “Down Trou”. \\ \hline
      If you're playing pool, you're playing pool & A phrase said to encourage a distracted player to give the game their full attention. \\ \hline
      Mishit & When the player strikes the cue ball badly. Often gives rise to \emph{The Clank}. Pronounced “miss hit”, rather than “mi-shit”.\\ \hline
      Moses & When the cue ball “parts the sea” of balls near a pocket and is the only ball that sinks. \\ \hline
      Rack 'em up! & A phrase said to the challenger to encourage them to rack the balls. \\ \hline
      Roll & Not all tables are perfectly flat, some have some deformations that can cause a change in a ball's path. This creates additional thrills and challenges for the players\dots and often home table advantage.\\ \hline
      Set & Overs (balls 9-15) or Unders (balls 1-7). \\ \hline
      Snooker & The player is unable to see either side of any of their balls from the cue ball. If the only one of their balls the player can see is touching the cue ball and would require a push shot, this is also a snooker.\\ \hline
      Snooker, Foul & A snooker that has come about as the result of a foul. \\ \hline
      Snooker, Legal & A snooker with no foul associated with it (i.e. performed legally), and has been done to the player by the opponent. \\ \hline
      Snooker, Partial & The player can only see one side of any of their balls. This includes the black/8-ball if the player has sunk all their colours. \\ \hline
      Snooker, Self & The player has snookered themself. \\ \hline
      Snooker, Table & The player is snookered because the cue ball is partially in a pocket. Hence, they are snookered by the table.\\ \hline
      Snooker, Total & See \emph{Snooker}. \\ \hline
      Tom Stick, The & The shortest cue available. Named after Tom Duncan who, at the time of coining the term, was the shortest player at the table. Tom would like it known that he is now over 6ft, and very rarely the shortest player. Recently this term has fallen out of use. \\ \hline
      White, The & Alternative term for the Cue Ball. \\ \hline
    \end{tabular}
\end{center}
\normalsize

\newpage
