

\placeframe{5}

\fancytitle[images/9ball.png]{9-Ball Pool}

\section{Object of the Game} \label{9ball:description}
% define "object balls" somewhere

\section{Racking} \label{9ball:racking}
\ex. \RackingUp[9ball]\par
\ex. \DetermineWhoRacks\par
\ex. \TableDiagram[9ball]\\
\fixedlabel{1.3.1}The nine numbered balls are racked in a triangle with the 1-ball placed in the front on the foot spot and the 9-ball in the centre (i.e., middle of the third row). There is no required pattern for balls 2-8.\itemspace
\fixedlabel{1.3.2}\TightlyPacked\itemspace
\fixedlabel{1.3.3}\RackPlacement[9ball]\itemspace
\fixedlabel{1.3.4}\RackingTool[9ball]\itemspace
\fixedlabel{1.3.5}\RackInspection[9ball]



\section{Breaking \& Push Out} \label{9ball:breaking}
\ex. The player breaking places the cue ball in the D.\par
\ex. The cue ball must first strike the 1-ball. Failure to do this is a foul.\par
\ex. \CueBallMissOffBreak[9ball]\par
\ex. \CueBallRailOffBreak\par
\ex. \CueBallSinkOffBreak\par
\ex. If any balls are knocked off the table during the break, this is a foul. See [INSERT HYPERREF] for returning the balls to the table.\par
\ex. If the player fouls the opponent is awarded ball-in-hand [INSERT HYPERREF].\par
\ex. If the player pockets one or more object balls (and does not foul) they may continue their turn. If they sink the 9-ball this is an instant win.\par
\ex. \label{9ball:pushout}If the player performs a legal break and sinks no object balls, they have the option of playing a Push Out.\itemspace 
\fixedlabel{2.9.1}Here the player may attempt to move the cue ball into a preferred position.\itemspace
\fixedlabel{2.9.2}The player must state their intention to attempt a push out. It is up to the opponent if failure to do so should result in a foul.\itemspace
\fixedlabel{2.9.3}The cue ball is not required to hit any ball or rail.\itemspace
\fixedlabel{2.9.4}If the 9-ball sinks it should be returned to the table [HYPERREF].\itemspace
\fixedlabel{2.9.5}If any object ball other than the 9-ball sinks it should not be respotted but the player may not continue their turn.\itemspace
\fixedlabel{2.9.6}Any foul results in ball-in-hand for the opponent as normal.\itemspace
\fixedlabel{2.9.7}Once the Push Out has been played the opponent may choose to either shoot from where the cue ball rests or to pass the turn back to the player.
\par

\section{General Play} \label{9ball:general}

\section{Fouling} \label{9ball:fouling}

\section{Ball in Hand} \label{9ball:ballinhand}

\section{Ending the Game} \label{9ball:ending}

\section{Etiquette and Behaviour} \label{9ball:etiquette}



%but if it does hit an object ball 






