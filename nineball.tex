\fancytitle[images/9ball.png]{9-Ball Pool}

\section{Object of the Game} \label{9ball:description}
% define "object balls" somewhere

\section{Racking} \label{9ball:racking}
\ruleitem \RackingUp[9ball]%
\ruleitem \DetermineWhoRacks%
\ruleitem \TableDiagram[9ball]%
\subruleitem The nine numbered balls are racked in a triangle with the 1-ball placed in the front on the foot spot and the 9-ball in the centre (i.e., middle of the third row). There is no required pattern for balls 2-8.%
\subruleitem \TightlyPacked%
\subruleitem \RackPlacement[9ball]%
\subruleitem \RackingTool[9ball]%
\subruleitem \RackInspection[9ball]%

\section{Breaking \& Push Out} \label{9ball:breaking}
\ruleitem The player breaking places the cue ball in the D.%
\ruleitem The cue ball must first strike the 1-ball. Failure to do this is a foul.%
\ruleitem \CueBallMissOffBreak[9ball]%
\ruleitem \CueBallRailOffBreak%
\ruleitem \CueBallSinkOffBreak%
\ruleitem If any balls are knocked off the table during the break, this is a foul. See [INSERT HYPERREF] for returning the balls to the table.%
\ruleitem If the player fouls the opponent is awarded ball-in-hand [INSERT HYPERREF].%
\ruleitem If the player pockets one or more object balls (and does not foul) they may continue their turn. If they sink the 9-ball this is an instant win.%
\ruleitem\label{9ball:pushout}If the player performs a legal break and sinks no object balls, they have the option of playing a Push Out.%
\subruleitem Here the player may attempt to move the cue ball into a preferred position.%
\subruleitem The player must state their intention to attempt a push out. It is up to the opponent if failure to do so should result in a foul.%
\subruleitem The cue ball is not required to hit any ball or rail.%
\subruleitem If the 9-ball sinks it should be returned to the table [HYPERREF].%
\subruleitem If any object ball other than the 9-ball sinks it should not be respotted but the player may not continue their turn.%
\subruleitem Any foul results in ball-in-hand for the opponent as normal.%
\subruleitem Once the Push Out has been played the opponent may choose to either shoot from where the cue ball rests or to pass the turn back to the player.%

\section{General Play} \label{9ball:general}

\section{Fouling} \label{9ball:fouling}

\section{Ball in Hand} \label{9ball:ballinhand}

\section{Ending the Game} \label{9ball:ending}

\section{Etiquette and Behaviour} \label{9ball:etiquette}



%but if it does hit an object ball 






