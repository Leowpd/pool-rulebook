\fancytitle[images/9ball.png]{9-Ball Pool}

\RuleSection{Object of the Game}{9ball:description} %\label{9ball:description}

% define "object balls" somewhere

\RuleSection{Racking}{9ball:racking} %\label{9ball:racking}

\ruleitem \RackingUp{9ball}%
\ruleitem \DetermineWhoRacks%
\ruleitem[9ball:rackpos]\TableDiagram{9ball}%
\subruleitem The nine numbered balls are racked in a triangle with the 1-ball placed in the front and the 9-ball in the centre (i.e., middle of the third row). There is no required pattern for balls 2-8.%
\subruleitem \TightlyPacked%
\subruleitem \RackPlacement{9ball}%
\subruleitem \RackingTool{9ball}%
\subruleitem \RackInspection{9ball}%

\RuleSection{Breaking \& Push Out}{9ball:breaking} %\label{9ball:breaking}

\ruleitem The player breaking places the cue ball in the D.%
\ruleitem The cue ball must first strike the 1-ball. Failure to do this is a foul. For fouls, see \ruleref{9ball:fouling}.%
\ruleitem \CueBallMissOffBreak{9ball}%
\ruleitem \CueBallRailOffBreak%
\ruleitem \CueBallSinkOffBreak%
\ruleitem If any balls are knocked off the table during the break, this is a foul. See [INSERT HYPERREF] for returning the balls to the table.%
\ruleitem If the player pockets one or more object balls:%
\subruleitem If the player also fouls then the player's turn ends and regular foul rules apply (\ruleref{9ball:fouling}).%
\subruleitem Otherwise the player may continue their turn (provided the 9-ball didn't sink).%
\subruleitem If the 9-ball sinks then this is an instant win for the player, provided no foul has occured.%
\subruleitem If the 9-ball sinks but the player fouls, the 9-ball should be returned to the table (see [INSERT HYPERREF] for respotting the 9-ball) and regular foul rules apply.%
\ruleitem[9ball:pushout]If the player performs a legal break and sinks no object balls, they have the option of playing a Push Out.%
\subruleitem Here the player may attempt to move the cue ball into a preferred position.%
\subruleitem The player should state their intention to attempt a push out.%
\subruleitem The cue ball is not required to hit any ball or rail.%
\subruleitem If the 9-ball sinks it should be returned to the table [HYPERREF].%
\subruleitem If any object ball other than the 9-ball sinks it should not be respotted but the player may not continue their turn.%
\subruleitem If the cue ball does hit an object ball, it must hit the 1-ball first. It is a foul otherwise.%
\subruleitem Any foul results in ball-in-hand for the opponent as normal.%
\subruleitem Once the Push Out has been played the opponent may choose to either shoot from where the cue ball rests or to pass the turn back to the player.%

\RuleSection{General Play}{9ball:general} %\label{9ball:general}

\ruleitem A player's turn consists of them striking the cue ball into the lowest numbered ball on the table, generally with the aim of:%
\subsubruleitem Sinking the 9-ball;%
\subsubruleitem Setting themselves up to sink the 9-ball;%
\subsubruleitem Snookering the opponent.%
\ruleitem During each shot, after the cue ball has hit the lowest numbered ball, at least one ball (object ball or cue ball) must hit the rail and/or an object ball must sink into a pocket. Failure to do this is a foul.
\ruleitem \FootOnGround%
\ruleitem If the player fouls, their turn ends and the opponent is awarded ball-in-hand (see \ruleref{9ball:fouling} for fouls and \ruleref{9ball:ballinhand} for ball-in-hand).%
\ruleitem If any object ball other than the 9-ball sinks, the player may continue their turn (provided no foul occured, otherwise regular foul rules apply).%
\ruleitem If the player sinks the 9-ball but also fouls on their shot, see [INSERT HYPERREF].%
\ruleitem \GenuineAttempt{9ball}%
\ruleitem \AlternateTurns{9ball}%
% ADD SECTION ON SNOOKERING

\RuleSection{Fouling}{9ball:fouling} %\label{9ball:fouling}

\ruleitem If a foul occurs, the player's turn ends and the opponent is awarded ball-in-hand (see \ruleref{9ball:ballinhand}).%
\ruleitem[9ball:ballinhand]Ball-in-hand:
\subruleitem Ball-in-hand allows you to pick up the cue ball and place it anywhere on the table.%
\subruleitem Once the cue ball is placed, the player may continue to adjust it until they are ready to take their shot.%
\subruleitem All balls must be completely at rest before the cue ball is permitted to be picked up. Touching the cue ball preemptively is a foul.%
\subruleitem The player must confirm they have been awarded ball-in-hand before picking up the cue ball. Touching the cue ball if the opponent did not foul is obviously a foul.%
\ruleitem \FoulsDontStack{9ball}%
\ruleitem \FoulCircumstances%
\subruleitem The player fails to hit the cue ball into lowest numbered ball first.%
\subruleitem During a Push Out, the cue ball hits any ball other than the 1-ball first (note that the cue ball may hit no ball).
\subruleitem After the cue ball has hit the lowest numbered ball, no ball (object ball or cue ball) hits a rail and no object ball sinks (this rule is excepted during a Push Out).%
\subsubruleitem If the cue ball or targeted object ball is touching a rail at the beginning of the shot, it is only counted as hitting a rail if it hits a different rail.%
\subruleitem \CueBallMiss\ (this rule is excepted during a Push Out).%
\subruleitem \CueBallSink%
\subruleitem \CueBallPreemptive%
\subruleitem The player picks up the cue ball for ball-in-hand before all balls are at rest.%
\subruleitem \ObjectBallTouch{9ball}%
\subruleitem \PushShot%
\subruleitem \BallOffTable%
\subsubruleitem If it is the cue ball then the opponent is awarded ball-in-hand.%
\subsubruleitem If it is an object ball other than the 9-ball, the ball is not returned to the table.%
\subsubruleitem If it is the 9-ball, the players attempt to place it as close to the foot spot as possible (see \ruleref{9ball:rackpos} for the position of the foot spot).%
\subruleitem \JumpShot%
\subruleitem \TableMovement%
\subruleitem \PoorBehaviour{9ball}%
\ruleitem[9ball:intentionalfoul]\IntentionalFoul{9ball}%

\RuleSection{Ending the Game}{9ball:ending} %\label{9ball:ending}

\ruleitem If the player legally (without fouling) sinks the 9-ball, the game ends and the player wins.%
\ruleitem \IntentionalFoulLoss{9ball}%
\ruleitem \MisleadingOpponentLoss{9ball}%
\ruleitem \PoorBehaviorLoss{9ball}%

\RuleSection{Etiquette \& Behaviour}{9ball:etiquette} %\label{9ball:etiquette}

\ruleitem[9ball:misleading]\Misleading{9ball}%
\ruleitem \DistractingWhileSettingUp{9ball}%
\ruleitem \DistractingWhileStriking{9ball}%
\ruleitem \Sportsmanship%
