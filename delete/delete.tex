\documentclass{article}

\usepackage{titlesec}
\usepackage{lipsum} % for dummy text
\usepackage{hyperref}


\newcommand{\itemspace}{\vspace{1.2mm}\\}

% === Counters ===
\newcounter{ruleitem}[section]
\renewcommand{\theruleitem}{\thesection.\arabic{ruleitem}}

\newcounter{subruleitem}[ruleitem]
\renewcommand{\thesubruleitem}{\thesection.\arabic{ruleitem}.\arabic{subruleitem}}

% === Ruleitem command (for 4.1., 4.2., etc.) ===
\newcommand{\ruleitem}{%
  \par\refstepcounter{ruleitem}%
  \setcounter{subruleitem}{0}% reset tertiary when a new rule starts
  \hangindent=3em
  \hangafter=1
  \noindent\makebox[3em][l]{\textbf{\theruleitem.}}%
}

% === Subruleitem command (for 4.1.1., 4.1.2., etc.) ===
\newcommand{\subruleitem}{%
  \par\refstepcounter{subruleitem}%
  \hangindent=5em
  \hangafter=1
  \noindent\makebox[5em][l]{\textbf{\thesubruleitem.}}%
}





\begin{document}

\section{General Play} \label{8ball:general}

\ruleitem A player's turn consists of them striking the cue ball into their coloured balls, typically with the goal of sinking them.
\ruleitem\label{test1}The player must strike the cue ball with at least one foot on the floor. “Tippy toes” are permissible.
\ruleitem If the opponent fouled on their turn, then the player gets an additional shot (see Section for more on fouls).
\subruleitem\label{test2}example example example example example example example example example example example
\ruleitem If a player sinks all of their coloured balls, they can now hit the black/8-ball with the cue ball directly.
\ruleitem The players alternate turns until the game is ended. See Section for the ending of the game.
\ruleitem See \hyperref[test1]{Section \ref*{test1}} and \hyperref[test2]{Section \ref*{test2}}

\end{document}
