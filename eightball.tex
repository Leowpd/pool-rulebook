\fancytitle[images/8ball.png]{8-Ball Pool}

\rulesection{Object of the Game}{8ball:description}

\ruleitem 8-Ball Pool is played between two players with a cue ball and fifteen object balls. Each player is assigned a set of seven balls. Balls 1-7 are “unders” and 9-15 are “overs”. Players take turns trying to sink their balls or disadvantage their opponent. Once a player sinks all their balls, they may attempt to sink the 8-ball and win.%

\rulesection{Format of Games}{8ball:format}

\ruleitem Games are generally played in a best-of-one, best-of-three, or best-of-seven format.%
\subruleitem Best-of-one matches are common when the players feel like a game but not too much of a commitment. This format sees whoever wins the game win overall.%
\subruleitem Best-of-three matches are the most common. The first player to win two games wins overall.%
\subruleitem Best-of-seven matches are usually the format for multi-day or holiday-long tournament-style competitions. In this, the first player to win four games wins overall. This format is more common in the Doubles variation of play (see \ruleref{8ball:doubles}).%

\rulesection{Racking}{8ball:racking}

\ruleitem \RackingUp{8ball}%
\ruleitem[8ball:challengerracks]\DetermineWhoRacks%
\ruleitem[8ball:rackpos]\TableDiagram{8ball}%
\subruleitem The fifteen numbered balls are racked in a triangle with the black/8-ball in the centre (i.e., middle of the third row). Other than the black/8-ball, there is no required pattern for the order of the balls.%
\subruleitem \TightlyPacked%
\subruleitem \RackPlacement{8ball}%
\subruleitem \RackingTool{8ball}%
\subruleitem \RackInspection{8ball}%

\rulesection{Breaking \& Establishment of Set}{8ball:breaking}

\ruleitem \CueBallPlacement%
\ruleitem The player breaking has the right to “call-the-break” (see \ruleref{8ball:callingthebreak}). This is optional. Remembering to do this is the player's responsibility.%
\ruleitem[8ball:breakingballs]The player breaking may hit the cue ball into any part of the rack, although hitting the front-most ball is typically recommended.%
\ruleitem \CueBallMissOffBreak{8ball}%
\ruleitem \CueBallRailOffBreak%
\ruleitem \CueBallSinkOffBreak%
\ruleitem If any balls are knocked off the table during the break, this is a foul. If it the black/8-ball, the player loses. See \ruleref{8ball:ballofftable} for returning the balls to the table.%
\ruleitem If the black/8-ball sinks or bounces off the table off the break, this is an instant loss for the player.%
\ruleitem If a coloured ball (or multiple coloured balls of the same set) sinks off the break, that set is the player's balls (i.e., overs or unders). The player can then take their next shot.%
\ruleitem If multiple coloured balls from different sets sink off the break, the player chooses which set to play. The player can then take their next shot.%
\ruleitem[8ball:breakingnoballs]If no balls sink off the break:%
\subruleitem At this stage, neither player has a set that belongs to them. The players will alternate turns to hit any coloured ball (not the black/8-ball) until a ball sinks and sets are assigned.%
\subruleitem If the black/8-ball is hit first, this is a foul.%
\subruleitem If multiple balls from different sets are sunk off the same shot, whichever went in first is the set assigned to the player. If neither player saw which ball sunk first, the player chooses their set. No foul is awarded and the player can continue their turn.%
\ruleitem If the black/8-ball sinks off the break, this is an instant loss.%

\rulesection{General Play}{8ball:general}

\ruleitem A player's turn consists of them striking the cue ball into their coloured balls, typically with the goal of sinking them.%
\ruleitem \FootOnGround%
\ruleitem If the opponent fouled on their turn, then the player gets an additional shot (see \ruleref{8ball:fouling} for more on fouls).%
\ruleitem If a player sinks all of their coloured balls, they can now hit the black/8-ball with the cue ball directly.%
\ruleitem If a player ever finds themselves in a snooker position, see \ruleref{8ball:snooker}.%
\ruleitem \AlternateTurns{8ball}%

\rulesection{Fouling}{8ball:fouling}

\ruleitem If a foul occurs, then the player's turn ends, and the opponent is granted an extra shot in their turn.%
\ruleitem \FoulsDontStack{8ball}%
\ruleitem \FoulCircumstances%
\subruleitem The cue ball is hit into a ball that is not the player's own first. This does not include coloured balls being hit during the break (\ruleref{8ball:breakingballs}) or during subsequent shots until a ball sinks (\ruleref{8ball:breakingnoballs}). Hitting the black/8-ball first remains a foul in these instances. This rule is excepted in some cases of a snooker (\ruleref{8ball:snooker}).%
\subruleitem \CueBallMiss%
\subruleitem \CueBallSink%
\subruleitem The player sinks an opponent's ball.%
\subruleitem \CueBallRail%
\subruleitem \CueBallPreemptive%
\subruleitem The player fails a nomination while attempting to escape some forms of a snooker (see \ruleref{8ball:snooker}).%
\subruleitem The cue ball is meant to be in the D, but the player strikes it while it is outside the D (reasonable allowances are made if it's only outside by a few milimetres).%
\subruleitem \PushShot%
\subruleitem \ObjectBallTouch{8ball}%
\subruleitem[8ball:ballofftable]\BallOffTable%
\subsubruleitem If it is the cue ball, the opponent takes their shot from the D.%
\subsubruleitem If it is a coloured ball, the players attempt to place it as close to the foot spot as possible (see \ruleref{8ball:rackpos} for the position of the foot spot). This also applies if multiple coloured balls are knocked off the table.%
\subsubruleitem If it is the black/8-ball, the player loses.%
\subruleitem \JumpShot%
%\subruleitem If the optional jump shot rule laid out in Section X.X is in place, illegal jump shots are considered a foul. An illegal jump shot is “scooping” the cue ball from underneath or if the cue touches the table during the shot.%
\subruleitem \TableMovement%
\subruleitem \PoorBehaviour{8ball}%
\ruleitem If a player is awarded a foul shot but then fouls on their first shot, then their turn ends and regular foul rules apply.%
\ruleitem[8ball:intentionalfoul]\IntentionalFoul{9ball}%

\rulesection{Snooker}{8ball:snooker}

\ruleitem A snooker (or “total snooker”) is defined as when the player is unable to see either side of any of their balls (if the only one of their balls the player can see is touching the cue ball and would require a push shot, this is also a snooker). The situation by which the snooker came to be may give rise to a nomination rule being enacted.%
\ruleitem A legal snooker is when the opponent has snookered the player legally.%
\subruleitem If the player is on their coloured balls, a snooker is a pain in the ass. The most common way of getting out of this is by using the cushions to rebound the cue ball.%
\subruleitem If the player is on the black/8-ball, they have the option of enacting \emph{Nomination Rule 1} (see \ruleref{8ball:nomination1}). This does not apply in the case of a self-snooker.%
\ruleitem A foul snooker is when the opponent fouled while snookering the player.%
\subruleitem The player has the option of enacting \emph{Nomination Rule 2} (see \ruleref{8ball:nomination2}).%
\subruleitem In the case of a foul table snooker (a table snooker being when the player is snookered because the cue ball is partially in a pocket), the player may move the cue ball to the D. If still snookered in the D, they may enact \emph{Nomination Rule 2} (\ruleref{8ball:nomination2}).%
\ruleitem In no case of self snooker is the player allowed to enact a nomination rule.%
\ruleitem A partial snooker is when a player can only see one side of any of their balls. The player may not enact a nomination rule.%
\ruleitem Nomination Rules:%
\subruleitem[8ball:nomination1]Nomination Rule 1: This rule occurs during a legal snooker on the black/8-ball. The player may nominate a ball that is not their own to hit first with the cue ball. This nominated ball must then hit the black/8-ball. The player must state the nominated ball verbally. If the cue ball hits any other ball before the nominated ball, it is a foul. If the nominated ball hits any other ball before it hits the black/8-ball, it is a foul. If the nominated ball misses the black/8-ball, it is a foul. If any ball sinks, it is a foul. This uses up one of the player's shots.%
\subruleitem[8ball:nomination2]Nomination Rule 2: This rule occurs during a foul snooker. The player may nominate any ball on the table for the cue ball to hit instead; the nominated ball is not required to hit the player's ball. The nomination must be stated verbally. If the cue ball hits any other ball before the nominated ball, it is a foul. The nominated ball may hit any or no ball. If any ball sinks, it is a foul. This uses up one of the player's shots.%
\ruleitem \GenuineAttempt{8ball}%

\rulesection{Special Rules}{8ball:special}

\ruleitem[8ball:callingthebreak]Calling-the-break. Before the break, the player breaking may call a specific ball they intend to sink off the break into a specific pocket. The player must verbally state the intended ball and pocket to the opponent. If the called ball goes into the called pocket, it is an instant win for the player. Otherwise, play continues as normal.\standardspace Any ball except the cue ball and the black/8-ball may be called.\standardspace If the called ball goes into the called pocket, but the cue ball also sinks, the player still wins.\standardspace If the called ball goes into the called pocket, but the black/8-ball also sinks or is knocked off the table, the player loses.%
\ruleitem Down Trou. If a player wins by sinking all their balls and the black/8-ball while the opponent has all seven of their balls still on the table, the opponent should drop their trousers and walk around the table in their undies with their pants around their ankles.\standardspace This may not always be appropriate, depending on the game's location and the people involved, it may be best to veto. Everyone involved must be comfortable for the down trou to take place.\standardspace If the opponent hasn't sunk any balls, but the player accidentally sinks one of theirs during the game, the opponent will have six (or fewer) balls remaining on the table. Therefore, this situation does not warrant a down-trou.\standardspace If the opponent has not sunk a single ball, but the game ends prematurely (i.e., the player wins through a method other than sinking all seven coloured balls and the black/8-ball), then this does not warrant a down trou.%
% \ruleitem Ball-in-Hand. When this rule is enacted, the player may pick up the cue ball and place it anywhere on the table. This rule is played only in rare circumstances, usually involving foul snookers and intentional fouls.%
% \ruleitem Jump Shots. The cue ball can be jumped if the player strikes it down into the table. This can be useful in getting out of snookers. When done on a low-quality table, this can damage the table, so this rule is often disallowed.%

\rulesection{Ending The Game}{8ball:ending}

\ruleitem If a player legally pockets the black/8-ball after sinking all seven of their coloured balls, the game ends, and the player wins.%
\ruleitem If a player fouls while on the black/8-ball, the game ends, and that player loses. See \ruleref{8ball:fouling} for fouls. This applies even after the 8-ball sinks (e.g., the black/8-ball sinks, but the cue ball keeps rolling and ends up sinking too).%
\ruleitem If a player prematurely (i.e., before all seven of their coloured balls are sunk) pockets the 8-ball, the game ends and that player loses.%
\ruleitem If the player breaking successfully “calls the break” (see \ruleref{8ball:callingthebreak} for “calling the break”) it is an instant win for the player.%
\ruleitem If, at any point, a player knocks the black/8-ball off the table, the game ends, and that player loses.%
\ruleitem \IntentionalFoulLoss{8ball}%
\ruleitem \MisleadingOpponentLoss{8ball}%
\ruleitem \PoorBehaviorLoss{8ball}%

\rulesection{Things that feel like cheating... but aren't}{8ball:notcheating}

\ruleitem Only playing for the snooker. When well behind, a player may no longer aim to sink their balls and instead resort to playing for a snooker. This may cause their opponent to foul while on the black and lose. This is not only allowed but encouraged!%
\ruleitem Jammy shots. These incredible feats of luck are probably some trickster god's will or something and we wouldn't want to defy that, so these are allowed.%

\rulesection{Doubles}{8ball:doubles}

\ruleitem Doubles is a game mode that allows for four players. It involves a slight alteration of the general play outlined in \ruleref{8ball:general}.%
\ruleitem The four players split into two teams of two.%
\ruleitem Both players on the “Challenger” team rack up the balls, and one player on the “Challenged” team breaks (see \ruleref{8ball:challengerracks} for how to determine which team racks).%
\ruleitem After the break, players from each team play alternate turns.%
\ruleitem Teammates are not allowed to switch turns during a game. Switching turns between games (e.g., in a best-of-three) is allowed.%

\rulesection{Etiquette \& Behaviour}{8ball:etiquette}

\ruleitem[8ball:misleading]\Misleading{8ball}%
\ruleitem \DistractingWhileSettingUp{8ball}%
\ruleitem \DistractingWhileStriking{8ball}%
\ruleitem[8ball:sportsmanship]\Sportsmanship%
